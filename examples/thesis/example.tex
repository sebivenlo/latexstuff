% This document is a minimal example for a thesis title and information page
% assumes that sebivenlo directory is in texmf tree
\documentclass[a4paper,twosided,11pt]{report}
\usepackage[ngerman,english]{babel}
\usepackage[utf8]{inputenc}
\usepackage[T1]{fontenc}
\usepackage[pdftex,scale={.8,.8}]{geometry}
\usepackage{layout}
\usepackage{fancybox}
\usepackage[pdftex]{graphicx}
\usepackage{fancyhdr}
\usepackage{%
  array,
  booktabs,
  dcolumn,
  rotating,
  shortvrb,
  units,
  url,
  lastpage,
  longtable,
  lscape,
  multirow,
  amssymb,
  amsmath,
  float,
  chngpage,
  colortbl,times,
}
\usepackage[toc,acronym]{glossaries}
\usepackage{sebicrccards}
\usepackage{sebirequirements}
\usepackage[left,pagewise,modulo]{lineno}
\usepackage{hyperref}
\setlength{\parindent}{0pt}
\setlength{\parskip}{.5\baselineskip}


%% ensure toc etc use roman numbers

\providecommand\Application{Provide an application name}
\providecommand\Customer{Provide a customer name}

\usepackage{metainfo}
\providecommand{\MineOnlyStart}{}
\providecommand{\MineOnlyEnd}{}

% environment for a two column table with proper alignment
% usage
% \begin{infoblock}
% Property: & Info
% \end{infoblock}
\newenvironment{infoblock}{
\begin{table}[h!]
\begin{tabular}{@{}p{0.25\textwidth}l}
}{
\end{tabular}
\end{table}
}

% Simple command for a fancy that attracts attention.
% usage:
% \LongQuote{the text you are quoting}{the name of the author}
% Any name can be passed (e.g. \citet{author2000text})
\newcommand{\LongQuote}[2]{
\begin{flushright}
	\begin{minipage}{.9\linewidth}
	\textit{#1}
	\end{minipage}

	\textit{#2}
\end{flushright}
}


\usepackage[
backend=biber,
style=authoryear,
natbib=true,
urldate=long,
dashed=false,
maxcitenames=3,
maxbibnames=10,
giveninits=true,
]{biblatex}

\DefineBibliographyStrings{english}{%
  bibliography = {References},
  urlseen = {Accessed},
  urlfrom = {Available at:}
}
\DeclareFieldFormat{urldate}{\mkbibbrackets{\bibstring{urlseen}\space#1}}
\DeclareFieldFormat{url}{\mkbibbrackets{online}\space\bibstring{urlfrom}\space\url{#1}}

% suppress qoutation marks from title
\DeclareFieldFormat[article,inproceedings]{citetitle}{#1}
\DeclareFieldFormat[article,inproceedings]{title}{#1}
\DeclareFieldFormat[article]{number}{\mkbibparens{#1}}

%% show name of second authors in references in form of lastname, initials
\DeclareNameAlias{sortname}{last-first}

% change journal format to volume(number) e.g. 1(4).
\renewbibmacro*{volume+number+eid}{%
  \printfield{volume}%
  \printfield{number}%
  \setunit{\addcomma\space}%
  \printfield{eid}}

% add comma after journaltitle
\renewbibmacro*{journal+issuetitle}{%
  \usebibmacro{journal}%
  \setunit*{\addcomma\space}%
  \iffieldundef{series}
    {}
    {\newunit
     \printfield{series}%
     \setunit{\addspace}}%
  \usebibmacro{volume+number+eid}%
  \setunit{\addspace}%
  \usebibmacro{issue+date}%
  \setunit{\addcolon\space}%
  \usebibmacro{issue}%
  \newunit}

% suppress "In" before journal name
\renewbibmacro{in:}{}

\usepackage{sebithesistitle}

\title{Thesis Title Goes Here}
\subtitle{Bachelor Thesis}
\author{Donald Duck}
\def\place{Place}
\date{\place, \today}

% provide data for information page
\def\documentname{Project	Plan for Bachelor Thesis Project}
\def\studentname{Donald Duck}
\def\snumber{1234567}
\def\course{Informatics - Software Engineering}
\def\period{February 2017 - July 2017}
\def\companyname{company abc}
\def\companyaddress{Some Address}
\def\companypostcodecity{12345, Somewhere}
\def\companycountry{Germany}
\def\companycoach{John Doe}
\def\companycoachmail{john.doe@company.com}
\def\universitytutor{Lecturer A}
\def\universitytutormail{lecturer.a@fontys.nl}
\def\examinator{Professional A}
\def\externalexpert{Professional B}
\def\hasnda{no}

\addbibresource{references.bib}

\begin{document}
\maketitle

\section*{Information Page}

Fontys Hogeschool Techniek en Logistiek\\
Postbus 141, 5900 AC Venlo

\vspace*{1cm}
\noindent
\documentname\

\vspace{1cm}

\begin{infoblock}
Name of student: & \studentname\\
Student number: & \snumber\\
Course: & \course\\
Period: & \period\\
\end{infoblock}

\begin{infoblock}
Company name: & \companyname\\
Address: & \companyaddress\\
Postcode, City: & \companypostcodecity\\
Country: & \companycountry\\
\end{infoblock}

\begin{infoblock}
Company coach: & \companycoach\\
Email: & \texttt{\href{mailto:\companycoachmail}{\companycoachmail}}\\
University coach: & \universitytutor\\
Email: & \texttt{\href{mailto:\universitytutormail}{\universitytutormail}}\\
\end{infoblock}

\ifx\examinator\empty
  \relax
\else
	\ifx\externalexpert\empty
		\relax
	\else
	  \begin{infoblock}
  	  Examinator: & \examinator\\
     External domain expert: & \externalexpert\\
     \end{infoblock}
	\fi
\fi


\begin{infoblock}
Non-disclosure agreement: & \hasnda
\end{infoblock}

\newpage
\phantomsection
\addcontentsline{toc}{chapter}{Statement of Authenticity}
\section*{Statement of Authenticity}
I, the undersigned, hereby certify that I have compiled and written the attached document / piece of work and the underlying work without assistance from anyone except the specifically assigned academic supervisors and examinors. This work is solely my own, and I am solely responsible for the content, organization, and making of this document / piece of work.

I hereby acknowledge that I have read the instructions for preparation and submission of documents / pieces of work provided by my course / my academic institution, and I understand that this document / piece of work will not be accepted for evaluation or for the award of academic credits if it is determined that it has not been prepared in compliance with those instructions and this statement of authenticity.

I further certify that I did not commit plagiarism, did neither take over nor paraphrase (digital or printed, translated or original) material (e.g. ideas, data, pieces of text, figures, diagrams, tables, recordings, videos, code, ...) produced by others without correct and complete citation and correct and complete reference of the source(s). I understand that this document / piece of work and the underlying work will not be accepted for evaluation or for the award of academic credits if it is determined that it embodies plagiarism.

\vspace*{1cm}

\begin{infoblock}
  Name: & \studentname \\
  Student Number: & \snumber \\
  Place/Date: & \place, \today
\end{infoblock}

\vspace*{1cm}

\begin{infoblock}
Signature: &
\end{infoblock}

\newpage
\chapter{Examples}
\section{Book}
\noindent
\verb=\citet{key}= results in \citet{goetz2006java}. \newline
\verb=\citep{key}= results in \citep{goetz2006java}. \newline
\verb=\cite{key}= results in \cite{goetz2006java}. \newline
\verb=\citep[p.123]{key}= results in \citep[p. 123]{goetz2006java}.

\section{Article}
\noindent
\verb=\citet{key}= results in \citet{cattell2011scalable}. \newline
\verb=\citep{key}= results in \citep{cattell2011scalable}. \newline
\verb=\cite{key}= results in \cite{cattell2011scalable}. \newline
\verb=\citep[p.123]{key}= results in \citep[p. 123]{cattell2011scalable}.

\section{Website}
\verb=\citet{key}= results in \citet{hom2017front}. \newline
\verb=\citep{key}= results in \citep{hom2017front}. \newline
\verb=\cite{key}= results in \cite{hom2017front}. \newline

\section{Quote}
\LongQuote{While better performance is often desirable -- and improving performance can be very satisfying -- safety always comes first. First make your program right, then make it fast -- and then only if your performance requirements and measurements tell you it needs to be faster.}{--- \citet[p. 221]{goetz2006java}}

\printbibliography
\end{document}
