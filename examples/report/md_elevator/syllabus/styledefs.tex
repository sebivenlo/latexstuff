% This is an annotated latex definition file.
% You can give it any name as long as its inputted somewhere
% in your code.
% You may even put it in a directory other then the current
% as long as that directory is in the TEXINPUTS path.
% Comments start with a % as you a seeing right now. Otherwise look
% to the left.
% font of the times family
\usepackage{nextpage}
\usepackage{times}
% make linenumbering possible for reviewing
\usepackage[pagewise,mathlines,displaymath]{lineno}
\usepackage[utf8]{inputenc}
% use the pdftex graphics extensions
\usepackage[pdftex]{graphicx}
%\DeclareGraphicsRule{*}{mps}{*}{}
\usepackage{picins}
\usepackage{ragged2e}
% geometry for the documents in this dir.
\usepackage[a4paper,includemp,
      scale={0.93,0.87},
      includeheadfoot,
      bindingoffset=1.2cm]{geometry}

%%% Local Variables: 
%%% mode: latex
%%% TeX-master: "sofa.tex"
%%% End: 


% to make the headers and footers look nice
\usepackage{color,fancyhdr}
% for version control info, e.g. in the header or footer
\usepackage{svn}
% make the header (and footer) span the whole printable area
\addtolength\headwidth\marginparwidth
\addtolength\headwidth\marginparsep
\pagestyle{fancy}
% to get version information in files, include the following lines in each document 
\newcommand\PartAuthor{Pieter van den Hombergh}
\newcommand\PartReviewer{Pieter van den Hombergh}
\SVN $Author: hom $
\SVN $Revision: 28 $
\SVN $Id: styledefs.tex 28 2014-09-24 09:04:24Z hom $
\SVN $Date: 2014-09-24 11:04:24 +0200 (Wed, 24 Sep 2014) $
\newcommand\TheFile{styledefs.tex}
%\renewcommand\footrulewidth{1pt}
% put revision control info in the footer
% this info comes from RCS but more likely from CVS
% the vspace compensates for the space taken up by the logo on the right hand.
% I like a fontys color in my header
\definecolor{fontys}{rgb}{0.2,0,0.2} % a darkish purple
\definecolor{purple}{rgb}{0.6,0,0.4} % a redish purple
\definecolor{navy}{rgb}{0.0,0,0.4} % a darkish blue
\renewcommand\headrule{%
  {\color{navy}%
    \hrule height 2pt width \headwidth
    \vspace{1pt}%
    \hrule height 1pt width \headwidth
    \vspace{-4pt}%
  }%
}%
\renewcommand\footrule{
  {\color{fontys}%
    \hrule height 1pt width \headwidth
    \vspace{1pt}%
    \hrule height 2pt width \headwidth
    \vspace{2pt}%
  }%
}
% do not indent pars, increase parskip
\setlength\parindent{0pt}
\setlength\parskip{0.5em}
% my own itemize: a bit less spacy then the default
\newenvironment{Itemize} {
  \begin{itemize}{}%
    \setlength\topsep{0ex}%
    \setlength\parskip{0ex}%
    \setlength\partopsep{0em}%
    \setlength\parsep{0em}%
    \setlength\itemsep{0em}%
    }%
  {\end{itemize}}
% make the head a bit heigher
\setlength\headheight{16pt}
\setlength\footskip{60pt}
% This macro '\define' puts the argument in em 
% and in boldface in the margin.
\newcommand{\Define}[1]{% 1 argument
  \mbox{}{\textit{#1}}% italics or em
  \marginpar{\raggedright% no adjust
    \bfseries\hspace{0pt}#1}% bold
} % end of macro
\newcommand\Okis[1]{{\color{fontys}\textbf{#1}}}
\newcommand\Code[1]{{\color{fontys}\texttt{#1}}}
\newcommand\Linux{{\color{fontys}\sc\bf Linux\ }}
% At start of pages, define sometimes puts the material in the wrong margin
% in that case use this fix.
\newcommand{\fixdefine}[1]{\mbox{}{\textit{#1}}%
  \reversemarginpar\marginpar{\bfseries\hspace{0pt}#1}\normalmarginpar}
% to display source code like things
\usepackage{listings}
\lstset{numbers=left} % to get out of the way with document line numbers
% to be able to 'on this page' and the like
\usepackage{varioref}
% to put a nice quote with the chapter
\usepackage[avantgarde]{quotchap}
\renewcommand\chapterheadstartvskip{\vspace*{-5\baselineskip}}
%select Helvetica for title and quote
\usepackage{helvet}
%\renewcommand\sectfont{\sffamily\bfseries}
%% example
%\begin{savequote}[10cm]
% \sffamily
%some text
%\end{savequote}
% needed for inclusion of gnumeric table
\def\inputGnumericTable{}
%% 
\usepackage{array}    
\usepackage{longtable}
\usepackage{calc}     
\usepackage{multirow} 
\usepackage{hhline}   
\usepackage{ifthen}   
%\usepackage[round]{jurabib}
\usepackage[square,sort,comma,numbers]{natbib}
\bibliographystyle{plainnat}
% the criteria a formatted as theorems
\usepackage{theorem}
%\usepackage{ragged2e,amsthm}
%\newtheoremstyle{ciss}%
%{5pt}{5pt}% space above below
%{\normalfont\RaggedRight}% style
%{-12pt}% indent
%{\sffamily\bfseries}{:}% heading font and punctuation
%{}% space after heading
%{}% head spec plain
\theoremstyle{margin}
{\theorembodyfont{\rmfamily}\newtheorem{Crit}{}[chapter]}
% Fancy headers are used
% To make the footer appear on chapter pages too,
% we redefine the fancypagestyle plain
\fancypagestyle{plain}{%
  \fancyhf{} % clear all  
  \fancyfoot[RO,LE]{\vspace{-1.2cm}%
  File: \TheFile\\%
  Author:\PartAuthor\\%
  Reviewer:\PartReviewer\\%
  Revision: \SVNRevision, \SVNDate\\
}
  \fancyfoot[RE,LO]{\includegraphics[width=2cm]{figures/fon000_00c.pdf}}
  \fancyfoot[C]{\bfseries\thepage}
  \renewcommand\headrule{}%
}% end of redef fancypagestyl plain
% Define my own fancy headers and footers  
\fancyhead{}
\fancyhead[RO]{\rightmark} % display section name
\fancyhead[LE]{\rightmark}
\fancyfoot[RO,LE]{\vspace{-1.2cm}%
  File: \TheFile\\%
  Author:\PartAuthor\\%
  Reviewer:\PartReviewer\\%
  Revision: \SVNRevision, \SVNDate\\
}
% logo in footer
\fancyfoot[RE,LO]{\includegraphics[width=2cm]{figures/fon000_00c.pdf}}
\pagestyle{fancy}
\usepackage{fancybox}
%\fancypage{}{\fbox}
%%%%%%%%%%%%%%%%%%%%%%%%%%
% For a watermark on each page:
\usepackage{eso-pic} % needed package
% Draw a DRAFT through the text.
% for other text renewcommand the next command
\newcommand\watermarktext{Draft}
% You could make this depend on the status of the files.
\makeatletter
  \AddToShipoutPicture{%
    \setlength{\@tempdimb}{.5\paperwidth}%
    \setlength{\@tempdimc}{.5\paperheight}%
    \setlength{\unitlength}{1pt}%
    \put(\strip@pt\@tempdimb,\strip@pt\@tempdimc){%
      \makebox(0,0){\rotatebox{45}{% for longer texts you might want to increase this angle
          \textcolor[gray]{0.85}{% the higher the number, the lighter the text
            \fontsize{8cm}{8cm}% Fit this font to the text
            \selectfont{\watermarktext}}}}
    }
}
\makeatother
%
% hyperef last
% this allows you to add hyperlinks in the document.
% It makes viewing it with e.g. Adobe acrobat very nice.
\usepackage{colortbl}
\usepackage{array}
\newcolumntype{I}{!{\vrule width 1pt}}
\newlength\savedwidth
\newcommand\whline{\noalign{\global\savedwidth\arrayrulewidth
    \global\arrayrulewidth 1pt}%
  \hline \noalign{\global\arrayrulewidth\savedwidth}}

\usepackage[pdftex,colorlinks=false,
                      pdfstartview=FitV,
                      linkcolor=blue,
                      citecolor=blue,
                      urlcolor=blue,
          ]{hyperref}
          \pdfinfo{
            /Title      (Sample latex document)
            /Author     (Pieter van den Hombergh 879417)
            /Keywords   (Elevator project Java latex documentation graphics math listings)
          }

\newenvironment{Enumerate}{
\begin{enumerate}
  \setlength{\itemsep}{1pt}
  \setlength{\parskip}{0pt}
  \setlength{\parsep}{0pt}}{\end{enumerate}
}

\newenvironment{Description}{
\begin{description}
  \setlength{\itemsep}{1pt}
  \setlength{\parskip}{0pt}
  \setlength{\parsep}{0pt}}{\end{description}
}

%% comments are inserted by emacs, that uses it to process
%% this latex file
%%% Local Variables: 
%%% mode: latex
%%% TeX-master: t
%%% End: 
