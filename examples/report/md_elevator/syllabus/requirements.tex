\renewcommand\TheFile{requirements.tex}
\SVN $Author: hom $
\SVN $Revision: 12 $
\SVN $Id: requirements.tex 12 2012-11-28 10:01:49Z hom $
\SVN $Date: 2012-11-28 11:01:49 +0100 (Wed, 28 Nov 2012) $
\SVN $State: Exp $
\begin{savequote}[8cm]
  \sffamily
  The risk is in the people. Or for the people.
  \qauthor{Anonymous}
\end{savequote}

\chapter{Requirements of an elevator system}
\section{Functional requirements}
\parpic(60mm,45mm)[rs]{\includegraphics[width=60mm]{figures/nurse_jackie_gal2_pr01_elevator.jpg}}This chapter describes some general requirements of an elevator
system.
 
The purpose of and elevator system is to transport goods or people in
an efficient and safe way between floors in a building. The system
may never cause any harm to its passengers or cargo.

An efficient system tries to minimise waiting time for the passengers,
either waiting for an elevator-cage to arrive at the floor she wants
to leave or waiting for the elevator-cage she is in to arrive at the 
desired floor.

The cages and cage shafts are grouped into shaft groups. The purpose of
the shaft groups is to coordinate the cage movement to improve the
provided transport service.

\subsection{Safety requirements}

The following requirements describe the safety regulations for the
system.
The order in the list is also the order of priority, highest priority first.
\begin{Enumerate}
\item The elevator system may never move the cage with open doors. The
  elevator door is considered open as long as the \textbf{door close 
  sensor} (or report) is not active.
\item The elevator system must (re)open its doors if the \textbf{obstruct
  sensor} is activated, unless the door is fully closed.
\item The elevator system must have an \textbf{alarm button} inside
  the cage that forwards an
  alarm signal to a service post that is always able to accept this
  call during the service hours of the elevator system. The response
  time must be less than .. minutes. Outside service hours the
  response time be less then  .. minutes.
\item The startup sequence must always result in a safe situation.
\item The shutdown sequence must always go through safe situations.

\end{Enumerate}

\subsection{Startup}
On startup the cage should move downward with its door(s) closed until
the lowest floor sensor is activated. Once the cage arrives at the
lowest floor, all requests are cancelled and the doors are opened. The
doors stay open as long as there are no up or down or target
requests. The startup sequence should obey the safety rules.


\subsection{Operation}
Normal operation of a cage starts at the lowest floor with the doors
open. The strategy to determine the movement  of the cages in a
multiple cage system should be such that the strategy optimises a
specific property of the system. This movement strategy should be
implemented in such a way that it is replaceable (Strategy Pattern). 

If there are no target requests for the cage nor up or down requests
from the system, we say that the cage is in the idle state.


The following section describes some strategies. This list is not exhaustive.

\subsubsection{Single cage strategies}
In any elevator system, the system services the requests in the order
a cage arrives at floors. There are two major modes:

\begin{Description}

\item[Full Pater Noster] Always make a complete circular movement
  between lowest and highest floor. That is: reverse direction of the
  elevator only at the top or bottom floor. The movement stops if
  there are no more requests in the forward circular direction. On
  arrival of request the movement is resumed in the same direction.
\item[Skipping Pater Noster] The direction may be reversed as soon as
  there are no more requests in the current direction. This avoids
  going to the extreme floors if there are no request from or to those
  floors. If there are no more requests or targets to visit, the cage
  can stay at the floor it is visiting.

  Example: The top floor as far as the elevator is consider is
  the roof of the building, which is seldom visited in daily use. The
  same goes for a cellar, which is floor zero as far as the elevator
  is considered. The normal entry to the building would then be on
  floor number 1.
\end{Description}

Skipping Pater Noster is the most used mode.

\subsubsection{Multiple cage strategies}
\begin{Description}
\item[Nurse mode] Think of a hospital. When a cage is put in nurse
  mode, that cage only obeys its target buttons. It should not service
  up and down request of floors. This mode can be turned on and off by
  means of a (key) lockable button inside the cage.
\item[Shortest travel time] This strategy tries to shorten the average travel
  and waiting time for all cages.
\item[Eager cage] In case of eager cage, the cage tries to pick up
  passengers as soon as possible.
\end{Description}


When implementing \textit{Shortest travel time} or \textit{Eager cage}
a \textbf{cost model} may be appropriate. A cost model computes some
virtual cost of an operation and tries to minimise that cost. In this
model the cost can be the respective times. Ingredients in the cost
model are travelling time between floors and estimated visit duration
on the floors to visit and of course the distance or number of floors
to travel.

\subsection{Shutdown}
On shutdown of a cage, all requests for that cage are cancelled and
the cage should \textit{stay at} the current floor or \textit{move to}
the nearest floor in the downward direction. 
On arrival on that floor the cage should open its door. After a 
transfer timeout the doors should be closed. During the shutdown state
of the entire system, the button lights should not react to up or down
requests. During the shut down state pressing any target button inside the
cage should trigger an alarm and reopen and then (after timeout) close
the doors.

\section{Non functional requirements}
For maintainability, and quality the following non functional
requirements have to be met:
\begin{Description}
  
\item[Package naming] All package names should start with the prefix
  \texttt{nl.fontys.sevenlo.prj32}
\item[Testing] All non graphical classes should be unit tested. Unit tests for
  all those classes are part of the artifacts in the repository.
\item[External lib storage] External libraries should be build in
  separate projects or be retrieved from their sources. Their source
  files should not be mixed with the application packages and
  files. Versions of used external jar files may be put into the
  repository although we recommend against this practice. 
\item[Graphical and sound resources] should be placed in the sources directory
  in a subdirectory named \texttt{resources}.
\item[Coding style] Use the java coding style as introduced in
  PRO2 (Java, semester 2). The style will be used on svn commit on all
  Java code using checkstyle\footnote{Version 5.5, use the appropriate
    netbeans checkstyle plugin} with the \texttt{fthv\_checks.xml}
  configuration file. The svn repository is configured to only accept
  java files that conform this coding convention.
  You can find this checkstyle configuration file in the trunk  of the
  the project svnroot \url{https://www.fontysvenlo.org/svn/2011/prj32m1/svnroot/trunk}. To check you style conformance beforehand you
  can install the checkstyle plug-in in netbeans, which flags all non
  conformance in the editor. 
\item[Code documentation] All classes and interfaces in the
  \texttt{src}-tree should be documented using javadoc. All members
  that have external visibility  (non private or have a getter) and
  all non private methods should have correct and complete javadoc
  documentation. For your package info files use the modern variant
  \texttt{package-info.java} in the packages. See the java doc
  documentation on how to write your javadoc. Note that javadoc
  conformance is also part of the coding convention.
\item[Settings and properties] To show various features of your
  product, use settings on the command line \texttt{-D}-option or
  property files liberally.
\item[Reporting] Write your documentation for the intended
  audience. Assume that the audience is knowledgeable in Java, UML and
  Patterns on your own level. Write concise (short) texts, to save your and
  my time. Keep it intelligible though. Add small diagrams to
  illustrate your story. Use Pattern names in your narrative, naming
  the participating classes with their role in the applied pattern.
  Use detailed diagrams only in the appendix. Maybe you could use this
  report (in \LaTeX\ format) as an example. You can find it's sources
  in the project repository.
\end{Description}

Example of the javadoc can be seen in code snippets \ref{snip:docclass}
and \ref{snip:docmethod}.

\lstset{basicstyle={\scriptsize},language=java,frame=shadowbox,rulesepcolor=\color{fontys}}

\newcommand\sourceroot{../sevenlowarrior/}

\lstinputlisting[firstline=7, lastline=20,framexleftmargin=8mm,%
caption={[Java doc for class]{Class javadoc
  example. From \texttt{.../IOWarriorConnector.java}}\label{snip:docclass}}]%
{\sourceroot/src/nl/fontys/sevenlo/iowarrior/IOWarriorConnector.java}

\lstinputlisting[firstline=92, lastline=106,framexleftmargin=8mm,%
caption={[Java doc for method]{Method javadoc
  example. From \texttt{.../IOWarriorConnector.java}}\label{snip:docmethod}}]%
{\sourceroot/src/nl/fontys/sevenlo/iowarrior/IOWarriorConnector.java}

\lstset{basicstyle={\normalsize}}

\section{No requirement at all}

In the organisation of our course, the students see this module at the
same time they learn fundamentals about algorithms and data
structures, in particular trees, lists and queues. For the students it
then seems natural to use this hammer to approach the elevator
problem, as in: put the requests in a queue and then deal with them by
searching and sorting for the right request to service next. 

The secret tip is: AVOID QUEUES of any kind in your design.  

What I\footnote{Author} learned in particular is that students tend to
build complex systems and than counteract design flaws with other
smart solutions.


