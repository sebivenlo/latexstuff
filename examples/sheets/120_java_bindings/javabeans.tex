\section[beans?]{What are Java Beans?}
\begin{frame}{What are \textit{Java Beans}}

  \begin{itemize}
  \item Alternate meaning: coffee beans of the java sort.
  \item In the java programming culture: Normal Java Classes that adhere to
    specific conventions:
    \begin{itemize}
    \item Must have public parameter-less (default) constructor. Other
      constructors are allowed.
    \item Have \textit{setters} and \textit{getters} for
      \textit{properties} with naming convention. 
    \item Can have property change listeners with support for them.
    \item Must be \textit{serializable}.
    \end{itemize}
  \item They often group or \textit{encapsulate} many objects into a
    single object.
  \end{itemize}
\end{frame}
\subsection[convention]{Naming convention}
\begin{frame}{Naming convention}
  All \textit{properties} of the bean should have getters and
  setters. For member named \Code{someValue} of \Code{SomeType}
  \begin{description}[get]
  \item[getter]\Code{getSomeValue}. First letter capitalized and
    prefixed with \Code{get} for getter: \Code{SomeType~getSomeValue()}
  \item[setter]\Code{setSomeValue}.  First letter capitalized and
    prefixed with \Code{set} for getter: \Code{void~getSomeValue(SomeType v)}
  \item[is] for \textit{getter} of boolean members: \Code{boolean~isSomeBoolean()}
  \end{description}
  If these conventions are followed, it is said that the
  implementation follows the Bean Pattern.

  This naming convention eases understanding, but also allows for
  automated tools to introspect the classes and objects and help with
  tools for non programmatic (as in GUI supported) operations on the
  \enquote{bean}. DEMO in Netbeans % use FaceBean.
  
\end{frame}

\subsection{Inspecting bean properties with a GUI}
\begin{frame}{Bean GUI in NetBeans}
  DEMO with FaceBean.

  Note that this version is slightly modified from the tutorial's
  version, using the inherited property change support of \Code{java.awt.Component}.
\end{frame}

\subsection[binding]{Property Binding}
\begin{frame}{Binding is like Observer}
  Binding means that a \textit{PropertyChangeListener} is informed
  whenever a property changes value.
  \begin{itemize}
  \item This is similar to the Observable/Observer pattern we saw
    earlier.

  \item \Code{PropertyChangeListener}s can be added and removed.
  \item The listeners are informed about the changes of the property by
    a call of their
    \Code{void~propertyChange(PropertyChangeEvent~evt)} method.

  \item When a bean wants to use this mechanism it typically uses a
    member of type PropertyChangeSupport, which  does the
    \enquote{Heavy Lifting}. In the FaceBean example \Code{mPcs}.
 
  \item The listeners are triggered by the bean by calling 
    \Code{mPcs.firePropertyChanged(propname,~oldValue,~newValue);}
  \end{itemize}

\end{frame}

\subsection[veto]{Suppress or reject changed: Veto}
\begin{frame}{Vetoable changes}
  A constrained property gets special treatment.
  \begin{itemize}
  \item The beans has veto change listeners.
    \begin{itemize}
    \item The listeners get a say if the change is allowed, based on
      name and old and new value of the property.
    \end{itemize}
  \item The FaceBean example uses \Code{VetoableChangeSupport} instance name \Code{mVcs}.
  \item The way the listener is informed of the immanent change is
    similar to firePropertyChange call on an earlier slide:
    \Code{fireVetoableChange(propname,~oldValue,~newValue)}
  \item However: call this method \alert{before} you change the
    properties value.
    \begin{itemize}
    \item Then when the listener vetoes the change, it will throw an
      exception, thereby typically preventing the execution of the
      steps to actually set the property to the vetoed value.
    \item This exception is \textit{checked}, so you must catch it
      or forward it with a \Code{throws} in the setter's signature.
    \end{itemize}
  \end{itemize}
\end{frame}
